%
% File acl2016.tex
%
%% Based on the style files for ACL-2015, with some improvements
%%  taken from the NAACL-2016 style
%% Based on the style files for ACL-2014, which were, in turn,
%% Based on the style files for ACL-2013, which were, in turn,
%% Based on the style files for ACL-2012, which were, in turn,
%% based on the style files for ACL-2011, which were, in turn,
%% based on the style files for ACL-2010, which were, in turn,
%% based on the style files for ACL-IJCNLP-2009, which were, in turn,
%% based on the style files for EACL-2009 and IJCNLP-2008...

%% Based on the style files for EACL 2006 by
%%e.agirre@ehu.es or Sergi.Balari@uab.es
%% and that of ACL 08 by Joakim Nivre and Noah Smith

\documentclass[11pt,a4paper]{article}
\usepackage[utf8]{inputenc}
\usepackage[T1]{fontenc}
\usepackage{acl2016}
\usepackage{times}
\usepackage{url}
\usepackage{latexsym}
\usepackage{booktabs}
\usepackage{relsize}

%\aclfinalcopy % Uncomment this line for the final submission
%\def\aclpaperid{***} %  Enter the acl Paper ID here

%\setlength\titlebox{5cm}
% You can expand the titlebox if you need extra space
% to show all the authors. Please do not make the titlebox
% smaller than 5cm (the original size); we will check this
% in the camera-ready version and ask you to change it back.


%% Greit å vite:

%% Siteringer:
%% Med parentes rundt hele: \cite{Gusfield:97}
%% Eller med parentes kun rundt år:  \newcite{Gusfield:97}
%% Flere på en gang: \cite{Gusfield:97,Aho:72}


%% Fra instruksjonene, om \aclfinalcopy :

%% By uncommenting {\small\verb|\aclfinalcopy|} at the top of this document, it
%% will compile to produce an example of the camera-ready formatting; by leaving
%% it commented out, the document will be anonymized for initial submission.  When
%% you first create your submission on softconf, please fill in your submitted
%% paper ID where {\small\verb|***|} appears in the
%% {\small\verb|\def\aclpaperid{***}|} definition at the top.

%% The review process is double-blind, so do not include any author information
%% (names, addresses) when submitting a paper for review.  However, you should
%% maintain space for names and addresses so that they will fit in the final
%% (accepted) version.  The ACL 2016 \LaTeX\ style will create a titlebox space of
%% 2.5in for you when {\small\verb|\aclfinalcopy|} is commented out.

%% \subsection{The Ruler}
%% The ACL 2016 style defines a printed ruler which should be presented in the
%% version submitted for review.  The ruler is provided in order that
%% reviewers may comment on particular lines in the paper without
%% circumlocution.  If you are preparing a document without the provided
%% style files, please arrange for an equivalent ruler to
%% appear on the final output pages.  The presence or absence of the ruler
%% should not change the appearance of any other content on the page.  The
%% camera ready copy should not contain a ruler. (\LaTeX\ users may uncomment
%% the {\small\verb|\aclfinalcopy|} command in the document preamble.)


\title{Optimizing a PoS Tag Set for Dependency Parsing}

\author{First Author \\
  Affiliation / Address line 1 \\
  Affiliation / Address line 2 \\
  {\tt email@domain} \\\And
  Second Author \\
  Affiliation / Address line 1 \\
  Affiliation / Address line 2 \\
  {\tt email@domain}  \\\And
  Third Author \\
  Affiliation / Address line 1 \\
  Affiliation / Address line 2 \\
  {\tt email@domain} \\}

\date{}

\begin{document}
\maketitle
\begin{abstract}
Abstraction ensues.
\end{abstract}

%\section{Credits}

\section{Introduction}
\label{sec:intro}
PoS tagging is an important preprocessing step for many NLP
tasks, such as parsing, named entity recognition and sentiment
analysis. Whereas much effort has gone into the development of
PoS-taggers, to the effect that this task is often considered
more or less a solved task, considerably less effort has been
devoted to empirical evaluation of the PoS tag sets
themselves. 
PoS tag sets are usually employed in a ``one size fits
all'' fashion, regardless of the requirements posed by the
downstream NLP tasks for which PoS-tagging is performed.

It is well known that syntactic parsing often benefits from quite
fine-grained morphological distinctions since morphology interacts
with syntax though phenomena such as agreement and case
marking. However, in a realistic setting where the aim is to
automatically parse raw text, the generation of morphological
information will often require a separate step of morphological
analysis that can be quite costly.

In this paper, we optimize a Norwegian PoS tag set for the task of
data-driven dependency parsing. In a set of experiments, various
morphological distinctions are introduced and evaluated both
intrinsically, i.e. in terms of PoS-tagging accuracy and
extrinsically, in terms of parsing accuracy.  Our results show that
the introduction of morphological distinctions not present in the
original tag set, whilst compromising tagger accuracy, actually leads
to significantly improved parsing accuracy. This optimization allows
us to bypass the additional step of morphological analysis, framing
the whole preprocessing problem as a simple tagging task.

The article is structured as follows. We start out by reviewing
previous work on tag set evaluation in Section \ref{sec:prev} and
\ref{sec:data} provides an overview of the treebank used for
experimentation. Section \ref{sec:setup} describes the experimental
setup used in our work and Section \ref{sec:optimization} goes on to
provide the results from our optimization experiments. Finally,
Section \ref{sec:summary} summarizes our main findings and discusses
some avenues for future work.

\section{Previous/Related work}
\label{sec:prev}
There is a considerable lack of previous efforts on extrinsic evaluation of the
effects of PoS tag sets on downstream applications. We will consider(/noe
annet, considerable...consider) papers that evaluate the effects of PoS tag
set granularity on PoS tagging. This includes investigation of the effects of
PoS tag sets on tagging of Swedish \cite{Meg:01,Meg:02} and English
\cite{Mac:05}. Increases in tagger accuracy does not readily translate to
improved parsing.

\cite{Meg:01,Meg:02} trained and evaluated a range of PoS taggers on the
Stockholm-Umeå Corpus (SUC) \cite{Gus:Har:06}, annotated with a Swedish version
of PAROLE tags, with the tag set totaling 139 tags. Furthermore, they
investigated the effects of tag set size on tagging by mapping the original tag
set into smaller subsets designed for parsing. They argue that a tag set with
complete morphological tags may not be necessary for all NLP applications, for
instance syntactic parsing. They found that the smallest tag set comprising 26
tags yields the lowest tagger error rate. However, for some of the taggers,
augmenting the tag set with more linguistically informative tags may actually
lead to a drop in error rate. They argue that this shows that the size of the
tag set as well as the type of information in the tags are crucial factors for
tagger performance. However/unfortunately, they do not report results of parsing
with the various PoS tag sets.

Similarly, \newcite{Mac:05} investigated the effects of PoS tag sets on PoS
tagger performance in English, specifically the Penn Treebank
\cite{Mar:San:Mar:93}. They mapped the original tag set of Penn Treebank to
more fine-grained tag sets "in terms of" linguistic utility. the additional
linguistic information included in the tags may assist the tagger.
utility/motivation to investigate whether linguistically motivated tag set
modifications can improve the tagger accuracy while increasing the linguistic
information in the tag set. Experimenting with both morphologically and
syntactically motivated modifications, they found that more fine-grained
tag sets rarely led to improvements in tagger accuracy,
where the most successful modification yielded an improvement in tagger
accuracy of 0.05 percentage points.

\section{Data}
\label{sec:data}
\subsection{The Norwegian Dependency Treebank}
We used the newly developed Norwegian Dependency Treebank (NDT)
\cite{Sol:Skj:Ovr:14}, which is the first publicly available treebank for
Norwegian. It was developed at the National Library of Norway in collaboration
with the University of Oslo, and contains manual syntactic and morphological
annotation. The treebank contains 311 000 tokens of Bokmål and 303 000 tokens
of Nynorsk.  The annotated texts are mostly newspaper text, but also include
government reports, parliament transcripts and excerpts from blogs. The
annotation process of the treebank was supported by the Oslo-Bergen Tagger
\cite{Hag:Joh:Nok:00} and then manually corrected by annotators.

%Linguistic motivation, insight, utility...

\paragraph{Morphological Annotation}
The morphological annotation and PoS tag set of NDT follows the Oslo-Bergen
Tagger \cite{Hag:Joh:Nok:00,Sol:13}, which in turn is largely based on the work
of \newcite{Faa:Lie:Van:97}. The tag set consists of 12 morphosyntactic PoS
tags, with 7 additional tags for punctuation and symbols.  The tag set is thus
rather coarse-grained, with broad categories such as \texttt{subst} (noun) and
\texttt{verb} (verb). The PoS tags are complemented by a large set of
morphological features, providing information about morphological properties such as
definiteness, number and tense. These features are used in our tag set
modifications, where the coarse PoS tag of relevant tokens are concatenated
with one or more of these features.  This allows us to include more linguistic
information in the tags.

\begin{table}
    \centering
    \smaller[0.5]
    \begin{tabular}{@{}ll@{}}
        \toprule
        \textbf{Tag} & \textbf{Description} \\
        \midrule
        \texttt{adj} & Adjective \\
        \texttt{adv} & Adverb \\
        \texttt{det} & Determiner\\
        \texttt{inf-merke} & Infinitive marker \\
        \texttt{interj} & Interjection \\
        \texttt{konj} & Conjunction \\
        \texttt{prep} & Preposition \\
        \texttt{pron} & Pronoun \\
        \texttt{sbu} & Subordinate conjunction \\
        \texttt{subst} & Noun \\
        \texttt{ukjent} & Unknown (foreign word) \\
        \texttt{verb} & Verb \\
        \bottomrule
    \end{tabular}
    \caption{Overview of the tag set of NDT.}
    \label{ndttagest}
\end{table}

\paragraph{Syntactic Annotation}
The syntactic annotation choices in NDT are largely based on the Norwegian
Reference Grammar \cite{Faa:Lie:Van:97}. The annotation choices are outlined in
Table \ref{ndtannotation}, taken from \newcite{Sol:Skj:Ovr:14}, providing
overview of the analyses of syntactic constructions that often distinguish
dependency treebanks, such as coordination and the treatment of auxiliary and
main verbs. The set of dependency relations contains 29 dependency relations,
including \textsc{adv} (adverbial), \textsc{subj} (subject) and \textsc{koord}
(coordination).

\begin{table} \centering
    %\smaller[0.5]
    \begin{tabular}{@{}ll@{}}
        \toprule
        \textbf{Head} & \textbf{Dependent} \\
        \midrule
        Preposition & Prepositional complement \\
        Finite verb & Complementizer \\
        First conjunct & Subsequent conjuncts \\
        Finite auxiliary & Lexical/main verb \\
        Noun & Determiner \\
        \bottomrule
    \end{tabular}
    \caption{Annotation choices in NDT.}
    \label{ndtannotation}
\end{table}

\section{Experimental Setup}
\label{sec:setup}
In preparation to conducting our experiments with linguistically motivated tag
set modifications, a concrete setup for the experiments needed to be
established, which is presented in the following.

\paragraph{Data Set Split}
As there was no standardized data set split of NDT due to its very recent
development, we needed to establish a data set split
(training/development/test) in preparation to our experiments. Our data set
split of the treebank follows the standard 80-10-10 (training/development/test)
split and will be distributed with the treebank and proposed as the new
standard(?). In creating the data set split, care has been taken to preserve
contiguous texts in the various data sets while keeping the split balanced in
terms of genre (and source). Our proposed data set split was used in the
Norwegian contribution to the Universal Dependencies project \cite{Ovr:Hoh:16}.
The split will be made available at a companion website.

\paragraph{Tagger}
For our experiments with tag set modifications, we wanted a PoS tagger that is
both fast and accurate. There is often a trade-off between the two, as the best
taggers tend to suffer in terms of speed due to their complexity. However, a
tagger that achieves both close to state-of-the-art accuracy as well as very
high speed is TnT \cite{Bra:00}. The fact that TnT was used for evaluating the
universal tag set \cite{Pet:Das:McD:12}, served as another good indication of
TnT being appropriate for our task. The sum of these factors led to TnT being
the tagger of choice for our experiments.

\paragraph{Parser}
In choosing a syntactic parser for our experiments, we considered previous work
on dependency parsing of Norwegian, specifically that of \cite{Sol:Skj:Ovr:14}.
They found the Mate parser \cite{Boh:10} to be the most successful parser for
the parsing of NDT. Furthermore, recent dependency parser comparisons
\cite{Cho:Tet:Ste:15} showed that Mate performed very well on parsing of the
English portion of the OntoNotes 5 corpus, beating a range of contemporary
state-of-the-art parsers.

\paragraph{Tag Set Mapping}
In order to alter the tag set of NDT, we created a mapping for carrying out the
tag set modifications. We created a mapping for carrying out the tag set
modifications that maps the relevant existing tags to new, more fine-grained
tags including more relevant morphological features for the applicable tokens.

\paragraph{Baseline}
It is common practice to compare the performance of PoS taggers to a
pre-computed baseline for an initial point of comparison.
For PoS tagging, a commonly used baseline is the Most Frequent Tag (MFT)
baseline, which we use in our experiments. This involves labeling each word
with the tag it was assigned most frequently in the training. All unknown
words, i.e., words not seen in the training data, are assigned the tag most
frequently assigned to words seen only once in the training. Unknown and
infrequent words have in common that they rarely occur, and we might therefore
expect them to have similar properties.
%, there are two main
%approaches: assign it the most frequent tag overall in the training data, or
%the tag most frequently assigned to words seen only once in the training data.
%We adopted the latter approach for our baseline tagger, as we believe that
%unknown words may have much in common with words that occur only once in the
%training, more so than simply words of the most frequent part-of-speech. The
%reason for this is that unknown and infrequent words have in common that they
%rarely occur, and we might therefore expect them to have similar properties.

\paragraph{Tags \& Features}
As we seek to quantify the effects of PoS tagging in a realistic setting, we
want to run the parser on automatically assigned PoS tags. For the training of
the parser, however, we have two options: using either gold standard or
automatically assigned tags. In order to settle on a configuration, we
conducted experiments with gold standard and automatically assigned tags to see
how they differ with respect to performance. The results of our experiments
reveal that the combination of training and testing on automatic tags is
superior to training on gold standard tags and testing on automatic tags,
surprisingly. Consequently, the parser was both trained and tested on
automatically assigned tags in our experiments.

Note that it is absolutely crucial that the morphological features in the
treebank are removed when using automatic tags, as they are still gold
standard. For instance, if a verb token is erroneously tagged as a noun, we
could potentially have a noun token with verbal features such as tense, which
markedly obfuscates the training and parsing. Another important factor is that
we want to isolate the effect of PoS tags, necessitating the exclusion of
morphological features.

\section{Tag Set Optimization/Experiments}
\label{sec:optimization}
With more fine-grained linguistically
motivated distinctions, we increase the linguistic information represented in
the tags, which may assist the tagger in disambiguating ambiguous and unknown
words, which in turn may aid the parser in recognizing and generalizing
syntactic patterns. However, the addition of more linguistic information to the
tags and thus a more fine-grained tag set will most likely lead to a drop in
tagger accuracy, due to the increase in complexity. The best tagging does not
necessarily lead to the best parse, and it is therefore interesting to
investigate how the tag set modifications may affect the interplay between
tagging and parsing.

\subsection{Tag Set Experiments}
We modified the tags for nouns, verbs, adjectives, determiners and pronouns in
NDT by appending selected sets of morphological features to each tag in order
to increase the linguistic information expressed by the tags.  For each tag, we
first experimented with each of the features in isolation before employing
various combinations of them. We based our choices of combinations on how
promising the features are and what we deem worth investigating in terms of
linguistic utility, in order to see how the features might interact.

\paragraph{Nouns}
In Norwegian, nouns are assigned gender (feminine, masculine, neuter),
definiteness (indefinite or definite) and number (singular or plural). There
is agreement in gender, definiteness and number between nouns and their
modifiers (adjectives and determiners).

\paragraph{Verbs}
Verbs are inflected for tense (infinitive, present, preterite, past perfect)
in Norwegian and can additionally take on mood (imperative, indicative) and
voice (active or passive).

\paragraph{Adjectives}
Adjectives agree with the noun they modify in terms of gender, number and
definiteness in Norwegian.

\paragraph{Determiners}
Like adjectives, determiners in Norwegian agree with the noun they modify in
terms of gender, number and definiteness.

\paragraph{Pronouns}
Pronouns in Norwegian include personal, reciprocal, reflexive and
interrogative. They can exhibit gender, number and person. Personal pronouns
have case (accusative or nominative).

%\begin{table}
    %\centering
    %\smaller[0.5]
    %\begin{tabular}{@{}lllll@{}}
        %\toprule
        %\textbf{Tag Set} & \textbf{MFT} & \textbf{Accuracy} &
        %\textbf{LAS} & \textbf{UAS} \\
        %\midrule
        %Original & \textbf{94.14\%} & \textbf{97.47\%} & 87.01\% & 90.19\% \\
        %Full & 85.15\% & 93.48\% & \textbf{87.15\%} & \textbf{90.39\%} \\
        %\bottomrule
    %\end{tabular}
    %\caption{Evaluation of tagging and parsing the development data with the two
        %initial tag sets.}
    %\label{inittagseteval}
%\end{table}

\begin{table*}
    \centering
    \smaller[0.5]
    \begin{tabular}{@{}llllll@{}}
        \toprule
        \textbf{Category} & \textbf{Feature(s)} & \textbf{MFT} &
        \textbf{Accuracy} & \textbf{LAS} & \textbf{UAS} \\
        \midrule
        \emph{Baseline} & --- & 94.14\% & 97.47\% & 87.01\% & 90.19\% \\
        Noun & Type, case \& definiteness & 89.61\% & 97.05\% & 88.81\% &
        91.73\% \\
        Verb & Finiteness & 93.72\% & 97.35\% & 87.30\% & 90.43\% \\
        Adjective & Degree & 94.13\% & 97.41\% & 87.29\% & 90.44\% \\
        Determiner & Definiteness & 94.13\% & 97.49\% & 87.30\% & 90.42\% \\
        Pronoun & Type \& case & 94.12\% & 97.51\% & 87.30\% & 90.41\% \\
        \bottomrule
    \end{tabular}
    \caption{Results of tagging and parsing with the most successful tag set
        modification for each category.}
    \label{respectiveresults}
\end{table*}

\begin{table}
    \centering
    \smaller[1]
    \begin{tabular}{@{}ll@{}}
        \toprule
        \textbf{Tag} & \textbf{Description} \\
        \midrule
        \texttt{adj|komp} & Comparative adjective \\
        \texttt{adj|pos} & Positive adjective \\
        \texttt{adj|sup} & Superlative adjective \\
        \texttt{det|be} & Definite determiner \\
        \texttt{det|ub} & Indefinite determiner \\
        \texttt{pron|pers} & Personal pronoun \\
        \texttt{pron|pers|akk} & Personal pronoun, accusative \\
        \texttt{pron|pers|nom} & Personal pronoun, nominative \\
        \texttt{pron|refl} & Reflexive pronoun \\
        \texttt{pron|res} & Reciprocal pronoun \\
        \texttt{pron|sp} & Interrogative pronoun \\
        \texttt{subst|appell} & Common noun \\
        \texttt{subst|appell|be} & Common noun, definite \\
        \texttt{subst|appell|be|gen} & Common noun, definite, genitive \\
        \texttt{subst|appell|ub} & Common noun, indefinite \\
        \texttt{subst|appell|ub|gen} & Common noun, indefinite, genitive \\
        \texttt{subst|prop} & Proper noun \\
        \texttt{subst|prop|gen} & Proper noun, genitive \\
        \texttt{verb|fin} & Finite verb \\
        \texttt{verb|infin} & Nonfinite verb \\
        \bottomrule
    \end{tabular}
    \caption{Overview of the optimized tag set.}
    \label{optimizedtagset}
\end{table}

\begin{table}
    \centering
    \smaller[0.5]
    \begin{tabular}{@{}lllll@{}}
        \toprule
        \textbf{Tag set} & \textbf{MFT} & \textbf{Accuracy} &
        \textbf{LAS} & \textbf{UAS} \\
        \midrule
        Original & \textbf{94.14\%} & \textbf{97.47\%} & 87.01\% & 90.19\% \\
        Full & 85.12\% & 93.46\% & 87.13\% & 90.32\% \\
        Optimized & 89.20\% & 96.85\% & \textbf{88.87\%} & \textbf{91.78\%} \\
        \bottomrule
    \end{tabular}
    \caption{Results of tagging and parsing with the optimized tag set, compared to
        the initial tag sets.}
    \label{finalresults}
\end{table}

\section{Optimized Pipeline}
\label{sec:pipeline}

\section{Summary/Conclusion and Future Work}
\label{sec:summary}

% \section*{Acknowledgments}

\bibliographystyle{acl2016}
\bibliography{references}
% \appendix
%\section{Supplemental Material}
%\label{sec:supplemental}

\end{document}
